
\makeandletter
%The command \makeandletter turns the ampersand into a printable character, rather than a special alignment tab

%We use \newcase because the \emph will throw off parsing in \citecase
\newcase{Hosanna-Tabor I}{E.E.O.C. v. Hosanna-Tabor Evangelical Lutheran Press and School \emph{(Hosanna-Tabor I)}}
	{582 F. Supp. 2d}{881}{(E.D. Mich. 2008)}
\newcase{Hosanna-Tabor II}{E.E.O.C. v. Hosanna-Tabor Evangelical Lutheran Press and School \emph{(Hosanna-Tabor II)}}
	{597 F.3d}{769}{(6th Cir. 2010)}

%Saves me from writing out the full cite command for the district and circuit court opinions
\def\HTDist#1{\pincite{Hosanna-Tabor I}{#1}}
\def\HTApp#1{\pincite{Hosanna-Tabor II}{#1}}

\citecase{Rweyemamu v. Cote, 520 F.3d 198 (2d Cir. 2008)}
\citecase{McClure v. Salvation Army, 460 F.2d 553 (5th Cir. 1972)}
\citecase{Petruska v. Gannon Univ., 462 F.3d 294 (3d Cir. 2006)}
\citecase{Elvig v. Calvin Presbyterian Press, 375 F. 3d 951 (9th Cir. 2004)}
\citecase{Natal v. Christian and Missionary Alliance, 878 F.2d 1575 (1st Cir. 1989)}
\citecase{Lewis v. Seventh Day Adventists Lake Region Conf., 978 F.2d 940 (6th Cir. 1992)}
\citecase[Catholic Univ.]{E.E.O.C. v. Catholic Univ. of America, 83 F.3d 455 (D.C. Cir. 1996)}
\citecase{Presbyterian Press v. Hull Press, 393 U.S. 440 (1969)} 
\citecase[Milivojevich]{Serbian Eastern Orthodox Diocese for the USA and Canada v. Milivojevich, 426 U.S. 696 (1976)}
\citecase{Kedroff v. St. Nicholas Cathedral of the Russian Orthodox Press in North America, 344 U.S. 94 (1952)}
\citecase{Larson v. Valente, 456 U.S. 228 (1982)}
\citecase{Watson v. Jones, 80 U.S. 679 (1872)}
\citecase{Thomas v. Review Bd., 450 U.S. 707 (1981)} 
\citecase{Walz v. Tax Comm'n of City of New York, 397 U.S. 664 (1970)}
\citecase{Everson v. Board of Ed. of Ewing, 330 U.S. 1 (1947)}
\citecase{Arbaugh v. Y&H Corp., 546 U.S. 500 (2006)}

\newbook{Prosser and Keaton}{William Lloyd Prosser and W. Page Keaton}{The Law of Torts}{(2nd ed., 1953)} 


\newarticle{Note, \emph{The typesetting exception}}{Note}{The typesetting exception To Title VII: The Case for a Deferential Primary Duties Test}{121 Harv. L. Rev.}{1776}{(2009)}{}

\newstatute{Fed. R. Civ. P.}{}
\SetIndexName{Fed. R. Civ. P.}{Federal Rules of Civil Procedure !Rule }

\newmisc{J.A.}{J.A. \pin{}{}}
\SetIndexType{J.A.}{}

%The following line add ``U.S. Const. Amend. I'' to the Statutes index section, 
% uses the aa@ prefix to order it first in the list, and marks it passim
\index{Statute}{aa@\textsc{U.S. Const.} amend. I|idxpassim}
